 %!TEX TS-program = xelatex
%!TEX encoding = UTF-8 Unicode

%\def \papersize {a5paper}
\def \papersize {a4paper}
%\def \papersize {letterpaper}

%\documentclass[14pt,\papersize]{extarticle}
\documentclass[12pt,\papersize]{extarticle}
% extarticle is like article but can handle 8pt, 9pt, 10pt, 11pt, 12pt, 14pt, 17pt, and 20pt text

\def \ititle {Origins of Mind: Lecture Notes}
\def \isubtitle {Lecture 01}
%comment some of the following out depending on whether anonymous
\def \iauthor {Stephen A.\ Butterfill}
\def \iemail{s.butterfill@warwick.ac.uk% \& corrado.sinigaglia@unimi.it
}
%\def \iauthor {}
%\def \iemail{}
%\date{}

%\input{$HOME/Documents/submissions/preamble_steve_paper4}
\input{$HOME/Documents/submissions/preamble_steve_lecture_notes}

%no indent, space between paragraphs
\usepackage{parskip}

%comment these out if not anonymous:
%\author{}
%\date{}

%for e reader version: small margins
% (remove all for paper!)
%\geometry{headsep=2em} %keep running header away from text
%\geometry{footskip=1.5cm} %keep page numbers away from text
%\geometry{top=1cm} %increase to 3.5 if use header
%\geometry{bottom=2cm} %increase to 3.5 if use header
%\geometry{left=1cm} %increase to 3.5 if use header
%\geometry{right=1cm} %increase to 3.5 if use header

% disables chapter, section and subsection numbering
\setcounter{secnumdepth}{-1} 

%avoid overhang
\tolerance=5000

%\setromanfont[Mapping=tex-text]{Sabon LT Std} 


%for putting citations into main text (for reading):
% use bibentry command
% nb this doesn’t work with mynewapa style; use apalike for \bibliographystyle
% nb2: use \nobibliography to introduce the readings 
\usepackage{bibentry}

%screws up word count for some reason:
%\bibliographystyle{$HOME/Documents/submissions/mynewapa} 
\bibliographystyle{apalike} 


\begin{document}



\setlength\footnotesep{1em}






%--------------- 
%--- start paste

      
\title {Presenting Your Research}
 
 
 
\maketitle
 
\subsection{title-slide}
I want to start with perhaps the simplest case of presenting your research ...
 
 
 
\section{What Is Your Thesis About?}
 
--------
\subsection{slide-3}
What do you say when someone asks, What is your thesis about??
Here's what I'd say.
 
It's about shared agency.  I provide a counterexample to the leading account of shared agency,
Michael Bratman's, and I show how a better account of shared agency can be provided by appeal to 
interagential structures of motor representation.
 
This question, What is your thesis about? should come up in several contexts:
\begin{enumerate}
\item when you attend conferences and workshops
\item when external speakers visit Warwick
\item when you talk to faculty and other graduate students
\item in conversation with non-philosophy friends, family and acquiantances
\end{enumerate}
 
A surprisingly common response is the wibble.
This is not typically going to inspire interest in your work.
So let's start by spending, say, 15 minutes trying to do something about this.
 
\subsection{slide-5}
Please think about what you would say.  You have three minutes to formulate a sentence or two.
You need to be able to say this in less than a minute.
I will then ask you to share this sentence with some people.
 
\subsection{slide-7}
[You are writing a sentence or two on what your thesis is about.]
 
\subsection{slide-8}
Now we're going to have an inner circle and an outer circle of people and the outer circle people
will ask the person opposite them in the inner circle, What is your thesis about?
The inner circle has less than 60 seconds to respond.
After 60s the inner circle will rotate clockwise by one person and we'll continue.
When we've been all the way around, we'll swap roles.
 
\subsection{slide-10}
[You are asking or answering the question, What is your thesis about?]
 
\subsection{slide-11}
What Is Your Thesis About?
... What should you say if someone asks you, What is your thesis about??
 
Presenting Your Research Online
... People should be able to find your email address, institutional affiliation and keywords about your research using google.
 
Giving Talks
... What is involved in presenting your research when giving a talk?
 
Writing an Abstract
... Many people will first encounter your research through an abstract, either for your thesis or else for a talk or publication.  What is the best way to use an abstract in presenting your research?
 
Presenting Your Research In Writing
... Presenting research in writing includes creating journal submissions, book chapters, drafts to circulate, conference submissions and your thesis.
 
\subsection{unit\_online}
True story.  I'm editing a special issue of a journal on your research area. I've seen that you've given a talk at a conference on this topic, and your abstract looks great. You're just the person I'm looking for to review a paper, and, if that goes well, invite to speak at a workshop. Can I find your email address and a short description of your research by googling? My experience is that, surprisingly often I can't.
 
 
 
\section{Presenting Your Research Online}
 
\subsection{slide-17}
True story.  I'm editing a special issue of a journal on your research area.
I've seen that you've given a talk at a conference on this topic, and your abstract looks great.
You're just the person I'm looking for to review a paper, and, if that goes well, invite to speak
at a workshop.
Can I find your email address and a short description of your research by googling?
My experience is that, surprisingly often I can't.
 
If you want to be part of an academic community that is wider than the circle of people you know 
and correspond with, you need a web page of some kind.
 
What does meeting this minimal requirement---being there---mean?
 
\subsection{slide-18}
You need a web page with ...
 
full name

            
email address

            
institution and position

            
two sentences about your research
 
This is the minimum you should provide.  
In addition, a photograph that enables people to tell whether you are the person they
met at whatever conference is a good idea.
 
Just out of interest, who currently satisfies this minimal requirement?
 
\subsection{slide-20}
What research should you present online?
 
\subsection{slide-22}
What's the most important thing?  Unless you have a strong publication, it's the thesis abstract.
 
\subsection{slide-26}
Should you use these things.  I have no idea, because I don't.
 
\subsection{unit\_conference}
I also want you to think more broadly than conferences.  Think, for example, about giving a work-in-progress talk at Warwick.  Or think about giving a job talk.  (I know there's a whole session on giving conference talks; there will probably be some overlap.)
 
 
 
\section{Giving Talks}
 
\subsection{slide-30}
When giving a talk, think about how many thinking hours you are consuming.
A 20 minute talk to 30 people consumes ten hours of other people's thinking time.
Other people's thinking hours are an extremely scarce resource.
Most of us are unlikely to get many.
 
\subsection{slide-32}
If you are going to consume so many thinking hours, you had better prepare meticulously.
 
\subsection{slide-35}
What does meticulous preparation mean?
Turn to your neighbour and spend two minutes reflecting on this.
 
\subsection{slide-37}
[You are discusing what is involved in preparing meticulously for a talk.]
 
I think the most important aspects of preparation concern the verbal performance.
I suggest these matter most:
\begin{enumerate}
\item Select your material (everything you say you have already thought through carefully, and discussed it with others ideally including your supervisor)
\item Be relevant (each talk is tailored to the particular audience)
\item Be awake (you haven't spent all night writing your talk, nor socializing)
\item Learn your lines (even if you are reading it out, you should know key formulations and be able to quote from memory)
\item Practice giving it  (practice the talk every day for seven days before, and practice similing at your audience while speaking) 
\item Know your timings (you won't know exactly how much time you have in advance, so be sure you can seamlessly cut material or speed up)
\end{enumerate}
 
\subsection{slide-38}
I want to emphasise the idea that you should practice giving you talk every day for 7 days before 
giving it.
Good preparation follows from this.
 
\subsection{slide-40}
Incidentally, you can easily tell whether someone has prepared meticulously for a talk.
If they say um while giving the talk, they have not prepared meticulously.
They are wasting your thinking hours.
 
\subsection{slide-41}
In presenting your research in a talk, the most important thing is the talk, of course.
But what about the supporting materials, your slides and handouts?
The first thing I want to say is that you should always both use slides and distribute a 
handout.
 
\subsection{slide-43}
Handouts are going out of fashion but I believe you should always create a handout.
Before I say why, let me ask you.
\textbf{What are the functions a handout can serve when you are giving a talk?}
 
There are two excellent reasons for using a handout.
 
The first is that they allow your audience to jump around different parts of your talk 
and to return to key claims and quotes.  
This is particularly useful in discussion.
Let me put it this way.
There is a tendency for discussion of a talk to focus on the last thing the speaker said.
This is probably because people can't easily remember things the speaker said further back in the 
talk.
You can improve the situation by providing a handout that helps people to recall material from
several parts of your talk.
 
Using a handout in this way will also facilitate better disucssion after the talk, either 
later in the day during social events or, sometimes, even days later.
 
The second reason for using a handout is this.
Everyone present has your name and email address in their hands.
You shouldn't underestimate the importance of this.  
For a typical audience with a typical speaker name, I'd guess that less than half will remember 
your name even 10 minutes into the talk.
And of course you want people to have your email address so they can ask you questions after.
 
So much for why you should use a handout.
What makes a good handout?
\begin{enumerate}
\item Your name and institutional email address are on it, of course.
\item It presents the structure of your argument
\item It contains quotes, claims and anything else the audience might want to ask about
\item It provides useful citations
\end{enumerate}
\textbf{Am I missing anything?}
 
One issue to consider with handouts is how much you are prepared to put on them.
At one extreme, 
some people provide multi-page handouts that are like a compressed version of the paper they're 
giving.
When I get these handouts, I usually spend a fair bit of the talk not listening but reading the 
handout.   I also usually read ahead of the speaker.
The other extreme is just to put quotes, citations and key claims onto the handout.
When I get these handouts, I usually write quite detailed notes in the talk in which I try to 
capture the structure of the talk and its main arguments.
I'm torn between these two extremes; I associate the more extensive handouts with better 
presentations, but I also like the idea that in attending a talk I'm focussed on the speaker and 
not just reading a papers.
\textbf{Help me out here.  What do you think?}
 
\subsection{slide-46}
I've just been saying that 
the primary function of a handout is to allow the audience to understand the structure and central
points of your talk without actually listening to what you say very carefully.
\textbf{Why use slides in presenting your research?}
 
Here are three reasons.
First, well-crafted slides will allow people who drift off to see whether you have moved on to a 
new point or not.
(Look at this slide and ask youself, Is he still waffling about handouts or has he moved on?  You can 
tell that I've moved on, surely?)
Second, slides allow you to underline things without making people search through a handout.
(You've seen a couple of examples of this in this talk, e.g. I used a slide to emphasise 
you should practice your talk for seven days before giving it.)
Third, using slides means that people are not looking at you all the time.
 
So here's my opinion.
You do not present your research on your slides, except where using diagrams or charts.
Rather your slides work in roughly the way that manual gestures do.
They provide emphasis at key points, and they indicate where one point ends and another begins.
\textbf{This is just one quite extreme opinion.  But I'm curious.  Who agrees with me about slides?  Are there more plausible views about the function of slides?}
 
So much for what slides are for.
What makes a good slide deck?
\begin{enumerate}
\item transitions between slides clearly convey transitions in the structure of the talk
\item the right points are emphasised
\end{enumerate}
\textbf{Am I missing anything?}
 
Before we move on, I also want to ask, What makes a bad slide deck?
(Bad handouts are typically the result of insufficient preparation, whereas it's possible to spend
arbitrarily long amounts of time preparing a truly atrocious slide deck.)
 
So what makes a bad slide deck?
\begin{enumerate}
\item the slide deck is the speaker's talk notes
\end{enumerate}
A lot of philosophers prepare a talk by writing sentences onto slides, and then show the slides
while they are talking.
This is terrible practice for at least three reasons.
First, your slides become a distraction from your talk.
Second, it makes your talk more boring than necessary.
Third, it's harder to read and listen simultaneously than to just do one or the other.
Your talk notes and your slides have completely different functions, so one cannot serve as the other.

You should never need to read your slides while giving a talk.
If you find yourself reading your slides, you have not prepared meticulously.

\textbf{Am I right?  Am I missing anything on what makes a bad slide deck?}
 
\subsection{slide-50}
So far we haven't talked about design.  
I want to encourage you not to worry about design too much.
For some people, cut and paste is something they do with scissors and glue.
I think this is fine.
I've seen beautiful handouts and I've seen handouts that have been literally glued together and
then photocopied.  As long as I can easily navigate through and read them, I think the design is not 
critical.
 
\subsection{slide-52}
Let me put it like this.
Meticulous preparation demands hours of your time.
So if you prepare meticulously, you probably don't have time for design.
Just try to avoid obvious mistakes:
\begin{enumerate}
\item print large and use generous margins
\item use a one font, colour and size 
\item never animate anything
\end{enumerate}
 
There are several talks I can remember for extraneous reasons; Joel Smith used a particularly nice
quality of paper for a handout at a talk at the joint session when we were graduate students,
and Gergo Csibra gave a talk here in Warwick with weird animated symbols.
But I can't remember the content of any of these talks.
 
\subsection{slide-53}
I've already said this but let me emphasise it as a common mistake.
 
\subsection{slide-54}
Simon suggested offering some practical advice.
 
\subsection{slide-55}
You *really* don't want to be the speaker wasting thinking hours trying to make the 
projector work.
 
\subsection{slide-57}
(a) The projector isn't working.  No worries, I've got handouts.
 
\subsection{slide-58}
(b) My laptop won't start.  No worries, I've got a usb.
 
\subsection{slide-59}
(c) My usb isn't working.  No worries, I emailed it to myself.
 
\subsection{slide-60}
(d) The internet isn't working either.  Lucky I learnt my lines.
 
\subsection{slide-62}
It won't help.  If you think you are going to waste my thinking hours, it doesn't help
me to tell this.
 
\subsection{slide-63}
The only time you should apologies is if you really expect the 
audience to leave.
 
\subsection{slide-64}
One last thing.
We've just been talking about how to use slides and handouts in presenting your research.
But the key thing in giving a talk is that you are there with the audience in one room
(or usually).  So ...
let your body express your ideas
 
This sounds like a funny thing to emphasise.  
But if you aren't doing this, should you be presenting your research by giving a talk at all?
 
(a) Make sure people can see your face and your hands.
(b) Don't have anything in your hands.
(c) Look at your audience.
 
\subsection{unit\_writing\_an\_abstract}
Many people will first encounter your research through an abstract, either for your thesis or else for a talk or publication.  What is the best way to use an abstract in presenting your research?
 
 
 
\section{Writing an Abstract}
 
\subsection{slide-66}
The answer depends on what is is an abstract of, of course.
 
\subsection{slide-67}
A thesis abstract might be read by (i) hiring committees, (ii) curious people in your research area,
and (iii) other students.
The abstract for a journal paper or conference talk will first be read by an editor and reviewers.
Later it will hopefully be read by people working in your area, and especially by people
wondering whether to read your paper or attend your talk.
 
I guess the thesis abstract has to be accessible to the widest possible audience, whereas
this is perhaps not desirable in the case of an abstract for a talk at a specialist conference or 
workshop; and a journal paper abstract probably falls between these extremes.
 
\subsection{slide-68}
What should an abstract convey?
\begin{enumerate}
\item your question
\item your key findings or conclusions
\item the structure of your work
\item where you are starting from
\item the quality of your writing (precision and readability)
\end{enumerate}
\textbf{Am I missing anything?}
 
\subsection{unit\_written\_presentation}
 
 
\section{Presenting Your Research In Writing}
 
\subsection{slide-70}
circulate your writing often, 
not too early, and 
mostly among friends
 
\subsection{slide-72}
Before submitting some writing to a journal certainly, and ideally to a conference as well,
you want feedback.
So you should circulate your work often.
To people you know and people you don't.  (The worst that can happen is that they won't read it.)
 
\subsection{slide-75}
But you should not normally circulate work that you already know how you can still improve it, except 
perhaps sometimes to your supervisor.
 
I think a good rule of thumb is this.  
Before circulating something, I have finished it on one day
and read the whole thing through, out loud, the next day.
Only then is it ready to circulate. (And, often enough, I find it isn't yet ready to circulate.)
 
Keep thinking hours in mind: the work you circulate should be meticulously prepared.
This is true even when sharing it among friends.
This includes:
\begin{enumerate}
\item the writing is precise and readable
\item it is well within any word limits
\item any guidelines are followed exactly (if a journal asked for double spaced, vertically oriented text in wordperfect format, do it!)
\item there are no typos or formatting errors
\item citations are present and complete 
\end{enumerate}
 
\subsection{slide-78}
Good friends read each others’ work.
 

%--- end paste
%--------------- 
 





\bibliography{$HOME/endnote/phd_biblio}



\end{document}